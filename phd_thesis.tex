% NOTE(mrksr): KOMA says about 60-70 letters per line and about 30 lines
% per page are desireable. Pretty much all theses have more so we also stay a
% bit above with about 85 letters and 40 lines for 17x24 and DIV=14. That's not
% so far from the standard typearea of a4paper with about 95 letters and 40
% lines.
\documentclass[
    paper=17cm:24cm,
    % paper=18cm:25.5cm,
    % paper=19cm:27cm,
    % a4paper,
    DIV=14,
    11pt,
    twoside,
    parskip=half,
    headsepline,
    toc=bibnumbered,
    listof=leveldown,
    captions=tableheading,
    chapterprefix=on,
    numbers=noendperiod,
]{scrbook}

% Standalone
\usepackage{currfile}
\usepackage{xstring} % standalone path magic

% Math
\usepackage{amsmath}
\usepackage{amssymb}
\usepackage{mathtools}
\usepackage{xfrac}

% Style
\usepackage[hypertexnames=false, unicode, pdfusetitle]{hyperref}
\usepackage[headsepline]{scrlayer-scrpage}
\usepackage{xcolor}
\usepackage{graphicx}
\usepackage[nobreak]{mdframed}

% Layout
\usepackage[font=small, labelfont=bf, format=plain, indention=1em]{caption}
\usepackage[skip=3pt]{subcaption}
\usepackage[standard,hyperref]{ntheorem}
\usepackage{booktabs}
\usepackage{tabularx}
\usepackage{colortbl}
\usepackage{siunitx}

% Navigation
\usepackage{url}
\usepackage[capitalise, nameinlink, noabbrev]{cleveref}
\crefformat{equation}{(#2#1#3)}
\usepackage[style=alphabetic, backend=biber, url=false, maxcitenames=1]{biblatex}

% Misc
\usepackage{pdfpages}
\usepackage{blindtext}

% TUM color theme
% Base Colors Corporate Design
\definecolor{tumblue}{HTML}{0065BD}
\definecolor{tumgreen}{HTML}{A2AD00}
\definecolor{tumorange}{HTML}{E37222}
\definecolor{tumivory}{HTML}{DAD7CB}
\definecolor{tumred}{HTML}{E53418} % not in Styleguide
\definecolor{tumviolet}{HTML}{69085A} % not in Styleguide

% Derived Colors
% https://kuler.adobe.com/create/color-wheel/
% https://portal.mytum.de/corporatedesign/print/styleguide/
% Grays - TUM
\definecolor{tumgray0}{HTML}{000000}
\definecolor{tumgray1}{HTML}{58585A}
\definecolor{tumgray2}{HTML}{9C9D9F}
\definecolor{tumgray3}{HTML}{D9DADB}
\definecolor{tumgray4}{HTML}{FFFFFF}

% Blues - TUM
\definecolor{tumblue0}{HTML}{003359}
\definecolor{tumblue1}{HTML}{005293}
\definecolor{tumblue2}{HTML}{0073CF}
\definecolor{tumblue3}{HTML}{64A0C8}
\definecolor{tumblue4}{HTML}{98C6EA}

% Greens - Adobe
\definecolor{tumgreen0}{HTML}{EAF900}
\definecolor{tumgreen1}{HTML}{AEBA00}
\definecolor{tumgreen2}{HTML}{8A9300}
\definecolor{tumgreen3}{HTML}{525800}

% Reds - Adobe
\definecolor{tumred0}{HTML}{F23719}
\definecolor{tumred1}{HTML}{CB2E15}
\definecolor{tumred2}{HTML}{90210F}
\definecolor{tumred3}{HTML}{65170B}

% Oranges - Adobe
\definecolor{tumorange0}{HTML}{F07824}
\definecolor{tumorange1}{HTML}{C9651E}
\definecolor{tumorange2}{HTML}{8E4715}
\definecolor{tumorange3}{HTML}{63320F}


% Siemens color theme
% Primary
\definecolor{sPetrol}{RGB}{0,153,153}
\definecolor{sStoneDark}{RGB}{60,70,75}
\definecolor{sStoneLight}{RGB}{135,155,170}
\definecolor{sStone}{RGB}{190,205,215}
\definecolor{sSandDark}{RGB}{115,100,90}
\definecolor{sSandLight}{RGB}{170,170,150}
\definecolor{sSand}{RGB}{215,215,205}
\definecolor{sSnow}{RGB}{255,255,255}

% Accent
\definecolor{sTealDark}{RGB}{0,100,110}
\definecolor{sTealLight}{RGB}{65,170,170}
\definecolor{sBlueDark}{RGB}{0,95,135}
\definecolor{sBlueLight}{RGB}{80,190,215}
\definecolor{sGreenDark}{RGB}{100,125,45}
\definecolor{sGreenLight}{RGB}{170,180,20}
\definecolor{sYellowDark}{RGB}{235,120,10}
\definecolor{sYellowLight}{RGB}{255,185,0}
\definecolor{sRedDark}{RGB}{100,25,70}
\definecolor{sRedLight}{RGB}{175,35,95}


% Custom colors
\definecolor{hannah0}{HTML}{d89d1c}
\definecolor{hannah1}{HTML}{d8801c}
\definecolor{hannah2}{HTML}{d8511c}
\definecolor{hannah3}{HTML}{d8bf1c}
\definecolor{hannah4}{HTML}{d8511c}


% We use precompiled images and do not add tikz for speed of compilation.
\newcommand{\includestandalonewithpath}[2][]{%
    \includegraphics[#1]{#2}
}
\newcommand{\includestandalone}[2][]{%
    \includegraphics[#1]{#2}
}
\endofdump
% Fonts
\usepackage{fontspec}
\usepackage{unicode-math}
\usepackage{libertinus}
\usepackage{microtype}

% Language
\usepackage[utf8]{luainputenc}
\usepackage{csquotes}
\usepackage[english, ngerman, main=english]{babel}
\selectlanguage{english}

% Sets
\newcommand{\Ab}{\mathbb{A}}
\newcommand{\Bb}{\mathbb{B}}
\newcommand{\Cb}{\mathbb{C}}
\newcommand{\Db}{\mathbb{D}}
\newcommand{\Eb}{\mathbb{E}}
\newcommand{\Fb}{\mathbb{F}}
\newcommand{\Gb}{\mathbb{G}}
\newcommand{\Hb}{\mathbb{H}}
\newcommand{\Ib}{\mathbb{I}}
\newcommand{\Jb}{\mathbb{J}}
\newcommand{\Kb}{\mathbb{K}}
\newcommand{\Lb}{\mathbb{L}}
\newcommand{\Mb}{\mathbb{M}}
\newcommand{\Nb}{\mathbb{N}}
\newcommand{\Ob}{\mathbb{O}}
\newcommand{\Pb}{\mathbb{P}}
\newcommand{\Qb}{\mathbb{Q}}
\newcommand{\Rb}{\mathbb{R}}
\newcommand{\Sb}{\mathbb{S}}
\newcommand{\Tb}{\mathbb{T}}
\newcommand{\Ub}{\mathbb{U}}
\newcommand{\Vb}{\mathbb{V}}
\newcommand{\Wb}{\mathbb{W}}
\newcommand{\Xb}{\mathbb{X}}
\newcommand{\Yb}{\mathbb{Y}}
\newcommand{\Zb}{\mathbb{Z}}

\newcommand{\Ac}{\mathcal{A}}
\newcommand{\Bc}{\mathcal{B}}
\newcommand{\Cc}{\mathcal{C}}
\newcommand{\Dc}{\mathcal{D}}
\newcommand{\Ec}{\mathcal{E}}
\newcommand{\Fc}{\mathcal{F}}
\newcommand{\Gc}{\mathcal{G}}
\newcommand{\Hc}{\mathcal{H}}
\newcommand{\Ic}{\mathcal{I}}
\newcommand{\Jc}{\mathcal{J}}
\newcommand{\Kc}{\mathcal{K}}
\newcommand{\Lc}{\mathcal{L}}
\newcommand{\Mc}{\mathcal{M}}
\newcommand{\Nc}{\mathcal{N}}
\newcommand{\Oc}{\mathcal{O}}
\newcommand{\Pc}{\mathcal{P}}
\newcommand{\Qc}{\mathcal{Q}}
\newcommand{\Rc}{\mathcal{R}}
\newcommand{\Sc}{\mathcal{S}}
\newcommand{\Tc}{\mathcal{T}}
\newcommand{\Uc}{\mathcal{U}}
\newcommand{\Vc}{\mathcal{V}}
\newcommand{\Wc}{\mathcal{W}}
\newcommand{\Xc}{\mathcal{X}}
\newcommand{\Yc}{\mathcal{Y}}
\newcommand{\Zc}{\mathcal{Z}}

% Random Variables
\newcommand{\rv}[1]{\symbf{#1}}
\newcommand{\map}[1]{#1^{\text{MAP}}}
% Matrix
\newcommand{\mat}[1]{\symbf{#1}}
\newcommand{\inv}{^{\raisebox{.2ex}{$\scriptscriptstyle-\mkern-1.5mu1$}}}
\newcommand{\tran}{^{\mkern-1.5mu\raisebox{.2ex}{$\scriptscriptstyle\mathsf{T}$}}}
\newcommand{\itran}{^{\raisebox{.2ex}{$\scriptscriptstyle-\mkern-1.5mu\mathsf{T}$}}}
\newcommand{\Eye}{\mat{\mathrm{I}}}
% Pseudo Inputs
\newcommand{\ps}[1]{\bar{#1}}
\newcommand{\psmat}[1]{\ps{\mat{#1}}}

% Nicer empty set
\renewcommand{\emptyset}{\varnothing}

% Math operators
% General
\DeclareMathOperator{\id}{id}
\DeclareMathOperator*{\argmax}{argmax}
\DeclareMathOperator*{\argmin}{argmin}
\DeclareMathOperator*{\argsup}{argsup}
\DeclareMathOperator*{\arginf}{arginf}
\DeclareMathOperator{\atanTwo}{atan2}
\DeclareMathOperator{\sgn}{sgn}
\DeclareMathOperator{\diag}{diag}
\DeclareMathOperator{\tr}{tr}
\DeclareMathOperator*{\maximize}{maximize}
\DeclareMathOperator*{\minimize}{minimize}
\DeclareMathOperator{\subjectto}{subject\ to}
\DeclareMathOperator{\Oh}{\mathcal{O}}
\DeclareMathOperator{\softmax}{softmax}
\newcommand{\Powerset}[1]{2^{#1}}
\newcommand*{\diff}{\mathop{}\!\mathrm{d}}
\DeclarePairedDelimiter{\abs}{\vert}{\vert}
\DeclarePairedDelimiter{\norm}{\Vert}{\Vert}
\newcommand{\nth}[2][th]{#2^{\text{#1}}}

% Probabilities
\DeclareMathOperator{\E}{\mathbb{E}}
\DeclareMathOperator{\cov}{cov}
\DeclareMathOperator{\var}{var}
\DeclareMathOperator{\corr}{\varrho}
\DeclareMathOperator{\p}{p}
\DeclareMathOperator{\q}{q}
\DeclareMathOperator{\KLdiv}{KL}
\DeclareMathOperator{\K}{\mathcal{K}}
\DeclareMathOperator{\Q}{\mathcal{Q}}
\DeclareMathOperator{\Norm}{\mathcal{N}\mkern2mu}
\DeclareMathOperator{\Multi}{\mathcal{M}}
\DeclareMathOperator{\Ber}{\mathcal{B}}
\DeclareMathOperator{\Uni}{\mathbb{U}}
\DeclareMathOperator{\Ind}{\mathbb{I}}
\DeclareMathOperator{\GP}{\mathcal{GP}}
\DeclareMathOperator{\ProbSpace}{\mathcal{P}\mkern1mu}
\DeclareMathOperator{\indep}{\perp\mkern-9.5mu\perp}
\DeclareMathOperator{\nindep}{\centernot{\indep}}

\providecommand\given{}
\DeclarePairedDelimiterX{\Cond}[1]{(}{)}{
    \renewcommand\given{%
        \nonscript\mkern2mu
        \delimsize\vert
        \nonscript\mkern2mu
        \mathopen{}
        \allowbreak}
    #1
}

\makeatletter
\newcommand{\Fun}{\@ifstar\@sfun\@fun}
\newcommand{\@fun}[1]{#1\Cond}
\newcommand{\@sfun}[1]{#1\Cond*}
\makeatother

\newcommand{\Probs}[1]{\ProbSpace_{#1}}
% \newcommand{\Probs}{\Fun*{\ProbSpace}}
\newcommand{\Prob}{\p\Cond}
\newcommand{\aProb}{\tilde{\p}\Cond}
\newcommand{\hProb}{\hat{\p}\Cond}
\newcommand{\Variat}{\q\Cond}
\newcommand{\Gaussian}{\Norm\Cond}
\newcommand{\Multinomial}{\Multi\Cond}
\newcommand{\Uniform}{\Uni\Cond}
\newcommand{\Indicator}{\Ind\Cond}

\DeclarePairedDelimiterX{\KLdelim}[2]{(}{)}{%
    #1\mkern2mu\delimsize\|\mkern2mu#2%
}
\newcommand{\KL}{\KLdiv\KLdelim}

\DeclarePairedDelimiterXPP{\Moment}[2]{#1}{[}{]}{}{
\renewcommand\given{%
    \nonscript\mkern2mu
    \delimsize\vert
    \nonscript\mkern2mu
    \mathopen{}
    \allowbreak}
#2
}

\providecommand\with{}
\DeclarePairedDelimiterX{\Set}[1]{\{}{\}}{
    \renewcommand\with{%
        \nonscript\mkern2mu
        \delimsize\vert
        \nonscript\mkern2mu
        \mathopen{}
        \allowbreak}
    #1
}

\DeclarePairedDelimiterXPP{\pix}[1]{\begingroup\scriptscriptstyle}{(}{)}{\endgroup}{\mkern-1mu#1\mkern-1mu}

% NOTE(mrksr): See https://tex.stackexchange.com/a/112212/34265
\makeatletter
\DeclareRobustCommand{\rvdots}{%
    \vbox{
        \baselineskip4\p@\lineskiplimit\z@
        \kern-\p@
        \hbox{.}\hbox{.}\hbox{.}
    }}
\makeatother


% Show overfull boxes
% \overfullrule=5pt

\addbibresource{zotero_export.bib}

\title{A unifying Bayesian formulation of Reinforcement Learning}
\author{
    Markus Kaiser\\
    \href{mailto:markus.kaiser@siemens.com}{\texttt{markus.kaiser@siemens.com}}
}
\makeatletter
\chead{\@title}
\makeatother

\setkomafont{labelinglabel}{\bfseries}

\begin{document}
\maketitle
\tableofcontents


\chapter{Structure from Composite Tasks}
\label{cha:structure_from_composite_tasks}
Prerequisites?
\begin{itemize}
    \item Measureable spaces
    \item Function Spaces
\end{itemize}

\section{Bayesian Probabilistic Numerics}
\label{sec:probabilistic_numerics}
\begin{figure}[t]
    \centering
    \includestandalone{figures/quantities_of_interest}
    \caption[Quantities of interest]{
        Quantities of interest.
        \label{fig:quantities_of_interest}
    }
\end{figure}
\begin{definition}[Numerical Method]
    \label{def:numerical_method}
    Let $\Uc$, $\Yc$ and $\Qc$ be measurable spaces and $Q: \Uc \to \Qc$ a measurable function.
    A \emph{numerical method} $(Y, A)$ for estimation of a quantity of interest $Q$ consists of
    \begin{labeling}{Information operator $Y$\quad}
        \item[Information operator $Y$] A measurable function in $\Uc \to \Yc$;
        \item[Estimation operator $A$] A measurable function in $\Yc \to \Qc$.
    \end{labeling}
\end{definition}

Just like everybody else, we use quadrature as a motivating example to show what quantities of interest are.
We assume $\Uu = \Fun*{C^0}{[a, b]; \Rb}$, the space of continuous real-valued functions on $[a, b]$.
The corresponding quantity of interest space is $\Qc = \Rb$ with the respective values given by the integral
\begin{align}
    Q(u) = \int_a^b u(t) \diff t
\end{align}
We choose $\Yc \subset \Rb^2$ with the information operator
\begin{align}
    Y_{\mat{x}}(u) = \left( \mat{x}, u(\mat{x}) \right)
\end{align}
which evaluates $u$ at the points $\mat{x} \in \Rb^N$ of our choice.
We can freely choose the vector $x_i$, giving rise to different information operators.
Here are a few well-known choices:
\paragraph{Monte Carlo integration}
Here, no explicit assumption about the form of $u$ are made.
\begin{align}
    \mat{x}
        &\sim \Uniform{[a, b]^N}
        &\Fun*{A_{\text{MC}}}{(\mat{x}, u(\mat{x})}
        &= \frac{1}{N} \sum_n u(x_n)
\end{align}

\paragraph{Trapezodial Rule}
The trapezodial rule approximates a function via a piecewise linear surrogate which is then integrated.
\begin{align}
    \mat{x}
    &= a + \frac{b - a}{N - 1} \cdot (0, 1, \cdots, N - 1)
    &\Fun*{A_{\text{TZ}}}{(\mat{x}, u(\mat{x})}
    &= \sum_{n=0}^{N - 1} \frac{b - a}{2} \left( u(x_{n-1}) + u(x_n) \right)
\end{align}

Note that all of these choices up to here are completely arbitrary, so none of those methods need to be particularly good.
They can, of cause, be motivated quite well.
We will show now how this motivation can be put in a probabilistic context.


\begin{figure}[t]
    \centering
    \includestandalone{figures/probabilistic_numerics}
    \caption[Probabilistic Numerics]{
        Probabilistic Numerics.
        \label{fig:probabilistic_numerics}
    }
\end{figure}
\begin{definition}[Probabilistic Numerical Method]
    \label{def:probabilistic_numerical_method}
    Let $\Uc$, $\Yc$ and $\Qc$ be measurable spaces and let $Q: \Uc \to \Qc$ be a measurable function.
    A \emph{probabilistic numerical method} $(Y, B)$ for estimation of a quantity of interest $Q$ consists of
    \begin{labeling}{Belief update operator $B$\qad}
        \item[Information operator $Y$] A measurable function in $\Uc \to \Yc$;
        \item[Belief update operator $B$] A measurable function in $\Probs{\Uc} \times \Yc \to \Probs{\Qc}$.
    \end{labeling}
\end{definition}

Explain Monte Carlo here?

\begin{definition}[Disintegration]
    Let $\Uc^y = \Set{u \in \Uc \with Y(u) = y}$.
    For $p \in \Probs{\Uc}$, a collection $\Set{p^y}_{y \in \Yc} \subset \Probs{\Uc}$ is a \emph{disintegration of $p$} with respect to a measurable $Y: \Uc \to \Yc$ if
    \begin{labeling}{Concentration\quad}
        \item[Concentration] $\Fun*{p^y}{\Uc \setminus \Uc^y} = 0$ for $Y_\#p$-almost all $y \in \Yc$;
        \item[Measurability] $y \mapsto p^y(f)$ is measurable;
        \item[Conditioning] $p(f) = \int p^y(f) Y_\#p(\diff y)$;
    \end{labeling}
    for each measurable $f : \Uc \to [0, \infty)$.
\end{definition}
This is an extension to conditionings of Random variables to the case where $\Uc^y$ is a null set.
Existence of disintegrations is guaranteed under weak conditions.

\begin{definition}[Bayesian Probabilistic Numerical Method]
    \label{def:bayesian_probabilistic_numerical_method}
    A probabilistic numerical method $M = (Y, B)$ is called \emph{Bayesian} for a quantity of interest $Q$ if for all $p \in \Probs{\Uc}$ and for $Y_\#p$-almost-all $y \in \Yc$
    \begin{align}
        \Fun*{B}{p, y} = Q_\#p^y.
    \end{align}
\end{definition}

Explain Trapezodial Rule here.


\section{Bayesian Optimization}
\label{sec:bayesian_optimization}
\emph{Crickets and Tumbleweed.}


\section{Reinforcement Learning}
\label{sec:reinforcement_learning}
\parencite{bertsekas_stochastic_1978}

\begin{definition}[Reinforcement Learning Problem]
    An \emph{infinite horizon stochastic reinforcement learning problem} is a seven-tuple $(\Sc, \Ac, \Zc, \Prob{z \given s, a}, f, \gamma, r)$ consisting of
    \begin{labeling}{Disturbance kernel $\Prob{z \given s, a}$\quad}
        \item[State space $\Sc$] A nonempty measurable space;
        \item[Action space $\Ac$] A nonempty measurable space;
        \item[Disturbance space $\Zc$] A nonempty measurable space;
        \item[Disturbance kernel $\Prob{z \given s, a}$] A measurable function in $\Sc \times \Ac \to \Probs{\Zc}$;
        \item[System function $f$] A measurable function in  $\Sc \times \Ac \times \Zc \to \Sc$;
        \item[Discount factor $\gamma$] A positive real number;
        \item[Reward function $r$] A differentiable function in $\Sc \times \Ac \to \Rb \cup \Set{\pm \infty}$.
    \end{labeling}
\end{definition}
We remove the control constraint because nobody cares.


\subsection{Reinforcement Learning}
\label{sub:reinforcement_learning}
\begin{figure}[t]
    \centering
    \includestandalone{figures/agent_environment_interaction}
    \caption[Agent-environment interaction]{
        The interaction between an agent and its environment in reinforcement learning happens at discrete time steps.
    }
    \label{fig:agent_environment_interaction}
\end{figure}
Reinforcement learning describes the general problem of learning to control a system by interaction in order to achieve a predefined goal.
An \emph{agent} has to decide on specific \emph{actions} to influence their \emph{environment}.
The boundary between agent and environment is shown in \cref{fig:agent_environment_interaction}.
They interact at specific discrete \emph{time steps} $t \in \Nb$.
At every such time step, the agent observes the environment via the \emph{state} $\mat{s}_t \in \Sc$, where $\Sc$ is the space of all possible states.
Based on this information, the agent has to decide which action $\mat{a}_t \in \Ac$ to perform.
The space of all possible actions $\Ac$ is assumed to be constant for all time steps and states.
The decision-process an agent employs in order to choose an action is called the agent's \emph{policy}.

\begin{definition}[Policy]
    A \emph{policy $\pi$} an agent follows encodes the choice it makes when faced with a decision.
    It is a function
    \begin{align}
        \pi: \Sc \to \Ac
    \end{align}
    which maps the current state of the system to the action the agent will perform.
\end{definition}

Once the agent has chosen an action for time step $t$, the state $\mat{s}_{t+1}$ is generated by the \emph{transition dynamics} $f$.
These dynamics are unknown to the agent and can contain probabilistic components such as noise.
However, the agent always observes one realization $\mat{s}_{t+1}$, the actual next state of the system.
This implies that given the same initial state $\mat{s}_0$ and policy $\pi$ it might be possible to generate different \emph{trajectories} $(\mat{s}_0, \dots, \mat{s}_T)$ by applying the transition dynamics $T$ times.
Together with the state and action spaces, the transition dynamics fulfill the Markov property.
That is, the distribution of $\mat{s}_{t+1}$ is independent of all states before $\mat{s}_t$ given $\mat{s}_t$.
\begin{definition}[Transition Dynamics]
    \label{def:transition_dynamics}
    The \emph{transition dynamics $f$} of a system encode its physical behaviour.
    These dynamics
    \begin{align}
        f: \Sc \times \Ac \to \Sc
    \end{align}
    stay constant over time but can be probabilistic.
\end{definition}

At time step $t+1$, the agent observes the state $\mat{s}_{t+1}$.
Additionally, it also receives a \emph{reward $r_{t+1}$}.
The reward is a quality assigned to the state transition from $\mat{s}_t$ to $\mat{s}_{t+1}$ using the action $\mat{a}_t$.
The higher the reward, the better the state transition is considered to be, independently of the future or past development of the system.
In the following, we consider $r_{t+1}$ to be independent of the state $\mat{s}_t$ and the action $\mat{a}_t$ and only depend on $\mat{s}_{t+1}$.
It is obtained from a real-valued and known \emph{reward function $r$} such that $r_{t+1} = r(\mat{s}_{t+1})$.
\begin{definition}[Reward Function]
    \label{def:reward_function}
    The \emph{reward function $r$} assigns a quality to each state in the state space
    \begin{align}
        r : \Sc \to \Rb.
    \end{align}
    This reward is the immediate feedback an agent receives when interacting with the system.
\end{definition}

The goal of the agent is to maximize the sum of all rewards earned while interacting with the system.
A greedy agent which is only concerned with the next immediate reward might not be the most successful, since it may be necessary to make a decision which is bad in the short term to gain an advantage in the long run, such as sacrificing a piece in chess to end up in a better position overall.

The \emph{value function} is a measure for how good a policy behaves in the long run.
Given a policy and an initial state, it is defined as the expected sum of rewards earned in a time horizon $T$.
Since the transition dynamics are assumed to be probabilistic, the states at all time steps greater than zero can be random variables.
Given a distribution of the state $\mat{s_t}$, the distribution for the next state $\mat{s_{t+1}}$ is
\begin{align}
    \Prob{\mat{s_{t+1}}} &= \int \Fun{f}{\mat{s_{t+1}} \given \mat{s_t}, \pi}\Prob{\mat{s_t}} \diff \mat{s_t},
\end{align}
where $f\Cond{\mat{s_{t+1}} \given \mat{s_t}, \pi(\mat{s_t})}$ denotes the probability of $\mat{s_{t+1}}$ under the distribution $f(\mat{s_t}, \pi(\mat{s_t}))$.

The time horizon $T$ can be chosen freely but does not depend on the policy or state.
The larger the time horizon, the more far-sighted an agent has to be to be successful.
For large values of $T$, it can be helpful to focus on rewards earned in the near future and to weight potential rewards further along with a smaller factor.
This is achieved with a constant discount factor $\gamma$.
\begin{definition}[Value Function]
    \label{def:old_value_function}
    Given transition dynamics $f$, a policy $\pi$, a time horizon $T \in \Nb \cup \left\{ \infty \right\}$ and a discount factor $0 \leq \gamma \leq 1$, the \emph{(expected) value function $J^\pi$} denotes the expected accumulated reward of a state and is given by
    \begin{align}
        J^\pi : \left\{
            \begin{aligned}
                \Sc &\to \Rb \\
                \mat{s} &\mapsto \Moment*{\E}{\sum_{t=1}^T \gamma^t r(\mat{s}_t) \given f, \pi, \mat{s}_t = \mat{s}}.
            \end{aligned}
        \right.
    \end{align}
    If the time horizon is infinite, $\gamma$ must be smaller than 1.
\end{definition}

Given a distribution of possible initial states $\Prob{\mat{s}_0}$, the sets $\Sc$ and $\Ac$ of states and actions together with the transition dynamics $f$, and the reward function $r$, reinforcement learning can be interpreted as a fully observable Markov decision process (MDP).
Note that extensions to partially observable MDPs (POMDPs) are regularly considered in the RL setting.

The objective of the reinforcement learning problem is to find the best policy in this decision process under the assumption that the transition dynamics are unknown a priori.
The \emph{optimal policy} $\pi^*$ maximizes the expected value under the distribution of initial states, that is it solves the optimization problem
\begin{align}
    \label{eq:optimal_policy}
    \begin{split}
        \pi^\ast &\in \argmax_{\pi} \Moment*{\E_{\Prob{\mat{s}_0}}}{J^\pi(\mat{s}_0)} \\
        &= \argmax_{\pi} \int J^\pi(\mat{s}_0) \Prob{\mat{s}_0} \diff \mat{s}_0.
    \end{split}
\end{align}


\subsection{Probabilistic Reinforcement Learning}
\label{sub:probabilistic_reinforcement_learning}
\begin{figure}[t]
    \centering
    \includestandalone{figures/quantities_of_interest_rl}
    \caption[Quantities of interest in reinforcement learning]{
        Quantities of interest in reinforcement learning.
        \label{fig:quantities_of_interest_rl}
    }
\end{figure}
\begin{table}[t]
    \centering
    \caption{Great Table!}
    \label{tab:label}
    \newcolumntype{Y}{>{\centering\arraybackslash}X}
    \begin{tabularx}{\textwidth}{cYY}
        \toprule
        & Probabilistic Numerics & Reinforcement Learning \\
        \midrule
        $\Uc$ & Latent function & True system-dynamics \\
        $\Qc$ & Definite Integral & Optimal value \\
        $\Yc$ & Function evaluations & Batch/Online data \\
        \midrule
        $Q : \Uc \to \Qc$ & Integration & Bellman principle \\
        $Y : \Uc \to \Yc$ & Observation & Exploration \\
        $B : \Yc \to \Qc$ & Quadrature & Policy search \\
        \bottomrule
    \end{tabularx}
\end{table}


\subsection{Bayesian Reinforcement Learning}
\label{sub:bayesian_reinforcement_learning}


\nocite{*}
\printbibliography
\end{document}
