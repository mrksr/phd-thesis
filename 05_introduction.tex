\chapter{Introduction}
\label{toc:introduction}
\begin{figure}[tp]
    \begin{subfigure}[b]{\halffigurewidth}
        \centering
        \includestandalonewithpath{intro_model_nature}
        \caption{
            \label{fig:introduction:models:nature}
            Process in nature
        }
    \end{subfigure}
    \hfill
    \begin{subfigure}[b]{\halffigurewidth}
        \centering
        \includestandalonewithpath{intro_model_black_box}
        \caption{
            \label{fig:introduction:models:black_box}
            Black-box model
        }
    \end{subfigure}\\[\figureskip]
    \begin{subfigure}[b]{\halffigurewidth}
        \centering
        \includestandalonewithpath{intro_model_white_box}
        \caption{
            \label{fig:introduction:models:white_box}
            White-box model
        }
    \end{subfigure}
    \hfill
    \begin{subfigure}[b]{\halffigurewidth}
        \centering
        \includestandalonewithpath{intro_model_structured}
        \caption{
            \label{fig:introduction:models:structured}
            Structured model
        }
    \end{subfigure}
    \caption[Comparison of model cultures]{
        \label{fig:introduction:models}
        In many machine learning problems, a model must be derived for a process in nature that is not fully understood.
        Pairs of observations $(\mat{x}, \mat{y})$ are available and the process is formalized as some functional dependence $\mat{y} = f(\mat{x})$.
        The black-box approach uses general function approximators to approximate the function $f$ given enough data but does not claim to yield insights into nature.
        The white-box approach uses scientific understanding to formulate a detailed model that explains the process in nature and uses data to find values for a small number of interpretable parameters.
        A structured model uses insights from both approaches and accepts that some processes cannot be modeled in full detail.
        Some structure is formulated based on scientific understanding and the datils are explained with general approximators trained with the available observations.
    }
\end{figure}
In many machine learning problems, the task is to derive a model for observational data that has been generated by some process in nature that is not fully understood.
Starting with an input-vector~$\mat{x}$ describing its initial state, nature produces an output-vector~$\mat{y}$ as shown in \cref{fig:introduction:models:nature}.
Given a set of observational pairs $(\mat{x}, \mat{y})$, we want to learn about the underlying, possibly stochastic, functional dependency $\mat{y} = f(\mat{x})$.
The goal of learning is often to either gain new knowledge about nature by describing the data or to be able to predict outputs $\mat{y}_\ast$ for previously unseen inputs $\mat{x}_\ast$.

\Textcite{breiman_statistical_2001,shmueli_explain_2010} discuss two approaches toward these goals using black-box (algorithmic) models and white-box (data) modeling cultures.
Traditional statistics favor white-box models which place a focus on describing and understanding observational data~\parencite{andrew_gelman_bayesian_2013,casella_statistical_2002,cox_planning_1958}.
In this worldview, data analysis is based on a stochastic model of nature that can explain how the data has been generated in great detail.
Such a model typically has a small number of unknown parameters whose values are estimated from the data.
\Cref{fig:introduction:models:white_box} shows how white-box models are typically formulated:
Based on the scientific understanding of nature, the process of interest is factorized into smaller components that are modeled separately and that interact to jointly explain the data.
Crucially, all components of a white-box model are designed by humans and are white-box as well to ensure an interpretable result.
Since white-box models represent a scientific understanding of nature, a model can be evaluated by how well it can describe the available observations:
A simple model with few parameters which is able to explain observations is expected to approximate the true process in nature reasonably well, thereby implementing Occam's razor~\parencite{thorburn_occams_1915}.
White-box models have been the driving force behind many advances in economics, medicine, and the natural sciences and allowed researchers to find and understand the structure in complex and heterogenous data~\parencite{efron_modern_2005,giulio_d._agostini_bayesian_2003}.
However, as processes get more complex and the amount of available data grows, formulating white-box models becomes a very difficult task~\parencite{sutton_bitter_2019}.

The black-box culture is more closely associated with machine learning~\parencite{mitchell_machine_1997,goodfellow_deep_2016}.
When formulating a black-box model, we accept that we cannot model nature directly and instead focus on the functional dependency $f$ between inputs and outputs as shown in~\cref{fig:introduction:models:black_box}.
Since one can show that many algorithms including neural networks or Gaussian processes are general approximators for smooth functions~\parencite{rasmussen_gaussian_2006}, given enough observations, they are expected to be able to approximate $f$ as well.
Since a sufficiently powerful approximator can explain any observation set, a black-box model can no longer be evaluated by how well it explains the training data.
Many functional dependencies can be used to generate a finite set of observations and since a black-box model does not have a scientific basis, it is hard to judge if the true process has been uncovered.
Instead, the performance of a black-box model is typically measured in terms of its ability to predict a test set of previously unseen data.
With growing computational power and the availability of large data sets, black-box approaches have led to impressive advancements in domains that are hard to formalize such as image recognition tasks~\parencite{lecun_backpropagation_1989} or playing the game of Go~\parencite{silver_mastering_2016}.
Since black-box models do not claim to uncover or represent knowledge about the true process in nature, their internal structure is generally not human-interpretable and their behavior is not well-understood.
As a result, their applicability in critical systems in the physical world is often limited.

We will study what we call structured models.
Structured models exist between these extremes as they are designed based on our scientific understanding of the underlying problem.
At the same time, we accept that there are problem-components we do not or cannot understand and use general function approximators to model these components.
\Cref{fig:introduction:models:structured} shows a structured model that has internal structure inspired by the more complex white-box model in~\cref{fig:introduction:models:white_box} but which also contains black-box components.
The goal of the structured modeling approach is to be more data-efficient compared to black-box models because there is less knowledge that needs to be inferred from data.
At the same time, imposing structure allows us to characterize desired solutions and formulate what we want to learn from data.
As a result, structured models can be formulated to gain insights about poorly understood system components from data.

Many successes in machine learning are based on structured models.
A well-known example is the convolutional neural network (CNN) used in computer vision tasks~\parencite{lecun_backpropagation_1989}.
Instead of, for example, using a standard multi-layer perceptron with sigmoid activation functions, the structure of a CNN is inspired by high-level knowledge about vision tasks.
Recognizing the importance of edge-detection and scale-invariance, a CNN consists of a stack of layers that implement a potentially large set of convolution kernels.
In contrast to earlier (more white-box) approaches to vision such as SIFT-features~\parencite{lowe_object_1999}, the parameters of the convolutions are learned from the data, thereby introducing a black-box component.
Another application of structured models can be found in natural language processing.
Formulating a white-box model for human language is prohibitively difficult.
However, insights from formal language theory~\parencite{chomsky_aspects_2014} can be used to formulate abstract expectations about the interaction of words in any language.
These insights lead to concepts such as the long short-term memory (LSTM)~\parencite{hochreiter_long_1997} or the attention mechanism~\parencite{bahdanau_neural_2014,chorowski_attention-based_2015}.

These and similar methods have seen great success recently in a wide range of digital domains such as speech recognition~\parencite{chorowski_attention-based_2015}, computer vision~\parencite{russakovsky_imagenet_2015}, video games~\parencite{berner_dota_2019}, or machine translation~\parencite{johnson_googles_2017}.
In such domains, data is often abundant, which shifts the focus away from traditional statistical models to approaches that are highly adaptable and scalable.
At the same time, the consequences of mistakes tend to be mild, enabling fast prototyping and heuristic design choices that can be validated in the deployed systems using the performance metrics of interest.

However, bridging the gap to applications in critical systems in the physical world has proved challenging.
Problem domains like robotics or industrial control introduce a new set of requirements.
In industrial control, machine learning is used to find more efficient or safer control strategies not obvious to the engineers designing a machine~\parencite{hein_benchmark_2017}.
In such scenarios, relevant data for finding new strategies are often scarce, since the most valuable data such as when a machine will fail, or how it will behave in new situations, is never produced.
As applications become more safety-relevant, gathering data through exploration can be problematic due to the adverse consequences of failure.
As a consequence, machine learning systems are required to handle low data regimes and uncertainties about a system's true behavior.

If they are to operate in safety-critical areas, interact with people, or carry responsibility, machine learning systems must be robust, trustworthy, and assessable.
Besides optimizing for good average-case performance, a responsible system needs to reason about plausible worst-cases to be able to avoid them.
To ensure trust, ML-systems often need to be verified by domain experts before deployment and are used to test hypotheses or make impactful decisions, emphasizing a need for interpretability.
Models are required to incorporate and reproduce expert knowledge and make consistent predictions, while at the same time yielding new insights from data.

The highly adaptable and scalable models that lead to success in digital domains are not well-suited to be deployed in safety-critical environments.
Large neural networks employed suffer from low interpretability, requirements of large amounts of data and their inability to quantify uncertainties~\parencite{goodfellow_deep_2016}.
This work explores an alternative approach to formulating structured machine learning models based on probabilistic modelling that can satisfy the requirements of physical environments.

\section{Bayesian structured models with Gaussian processes}
A key technique that allows us to cope with the requirements of critical systems in the physical world is probabilistic modeling.
Bayesian inference offers a framework to formalize expert knowledge as in prior assumptions and combine them with potentially uncertain data to represent knowledge about a system.
Using Bayes' rule, a distribution of plausible models is reweighed by their ability to explain observed data to yield a posterior distribution over plausible explanations.
To make predictions, all plausible models are taken into account, allowing us to both quantify confidence in predictions and take more unlikely but relevant scenarios into consideration.

Bayesian reasoning was pioneered by Pierre-Simon Laplace in the eighteenth century.
Laplace extended the prevalent mechanistic world-view held by scholars like Newton with the idea that knowledge can be imperfect or incomplete.
Laplace used Bayesian inference to formalize scientific reasoning under uncertainty and to enable the integration of multiple sources of data of varying quality.
Based on white-box models grounded in scientific understanding, this reasoning for example led to improved estimates of the masses of Jupiter and Saturn and the discovery of the uneven proportion of male and female human births~\parencite{pulskamp_laplace_2020,laplace_theorie_1820}.
Modern science has moved beyond phenomena that can be explained by a deterministic world-view held by Newton and Laplace towards subjects where randomness plays a major role.
Large-scale experiments in fields such as particle physics~\parencite{khachatryan_observation_2015} or astronomical observations~\parencite{collaboration_first_2019,the_ligo_scientific_collaboration_observation_2016} uncover scientific knowledge hidden in a sea of random events.
While Bayesian statistics is successfully used to identify the signal in highly noisy measurements, this analysis is still based on white-box modelling approaches to ensure that results conform to the requirements of scientific rigor.

This dissertation explores how advances in black-box machine learning can be integrated in Bayesian reasoning to futher relax the mechanistic world-view and allow us to tackle problems where white-box models are not available.
For applications in the physical world, expert knowledge is often based on an intuitive understanding of the underlying physics leading to coarse expectations about system-behavior on different layers of abstraction.
However, complete Bayesian white-box models are often too complex to formulate or computationally intractable.
We will discuss how abstract knowledge can be exploited to formulate structured Bayesian models that reproduce expectations but are still able to learn from data.
We will explore two approaches to adding such a structure.
First, white-box mathematical structure can be relaxed to a structured hierarchical system to learn new insights from data.
And second, structure motivated by expert-understanding of an underlying physical process can be used to formulate semantic constraints to a data-centric black-box approach.

The black-box and white-box modelling cultures follow different approaches to evaluating models.
While a white-box model is evaluated through a combination of goodness-of-fit measures on observations used during inference and subjective measures based on interpretability, black-box approaches typically focus on performance measures on test-sets.
Since structured models combine white-box and black-box properties, evaluation strategies should be a combination of both approaches.
We will show that good predictive performance is often not sufficient to identify a desirable structured model.
To formalize the subjectiveness in the description of desirable models, we will explore how downstream tasks can be taken into account during model selection.


\section{Thesis outline}
In this thesis, we will study how to formulate and evaluate Bayesian structural models using methods of Bayesian nonparametrics.
\todoi{Why GPs?}
The content is structured as follows.
In \cref{toc:gp}, we provide a short introduction to Bayesian machine learning and Bayesian nonparametrics with a focus on Gaussian processes (GPs).
We discuss the connection between statistical learning theory and Bayesian inference and their application to hierarchical models.
After introducing GPs, we discuss variational approaches to sparse GPs, their generalization to deep GP models, and inference schemes.

In \cref{toc:data_association}, we explore how to relax white-box structure in a structured hierarchical model using the data-association problem as an example.
We use the formal mathematical structure to separate the data-association problem into separate hierarchical components and derive a joint inference scheme, resulting in a fully Bayesian model.
We show that this model is able to factorize multi-modal data into independent processes, providing explicit models for both the processes and assignment probabilities.

In \cref{toc:alignment}, we discuss how to introduce additional constraints arising from expert knowledge about the data generation process to black-box deep GPs.
We interpret the problem of modeling the power production of multiple wind-turbines in a wind-farm as a nonlinear alignment problem.
Based on expert knowledge about the underlying latent and turbulent wind field, we can derive a structural model that enforces a physically plausible dependency structure between multiple time-series.
We discuss how these constraints help to solve an otherwise highly ambiguous learning problem and show that the imposed structure leads to a rich internal model-structure.

In \cref{toc:interpretable_rl}, we consider how to formalize the subjectiveness introduced via expert knowledge and model ambiguities by incorporating it into the model itself.
We include the task a model will be used to solve into model selection, which allows us to distinguish between qualitatively different models showing similar performance metrics.
We revisit the data-association problem and embed it into a reinforcement learning problem, where identifying the correct underlying dynamics is critical to finding a successful policy.
We show that semantic hierarchical structure increases data efficiency and allows domain experts to influence agent behavior through detailed insights into the dynamics of a system.

\Cref{toc:discussion} concludes the work presented in this thesis.
In \cref{toc:discussion:composition}, we present an intuitive argument for why inference schemes based on factorizations between layers cannot represent heterogeneous posteriors.
In \cref{toc:discussion:bo}, we argue why models with suboptimal marginal likelihoods can perform well in hierarchical systems.
In \cref{toc:discussion:mountaincar}, we explore this idea further and consider how tasks can be included in the inference problem directly.
Finally, we discuss possible further directions of research.


\section{Contributions}
This work presents novel Bayesian structured models inspired by industrial applications.
In \cref{contrib:data_association}, we formulate a fully Bayesian interpretation of the data association problem.
\Cref{contrib:patent_depeweg,contrib:patent_geipel} discuss an application of this model to the prediction of the combustion dynamics of a gas-turbine.
We formulate a hierarchical Bayesian model for a nonlinear alignment problem that is capable of representing the complex interactions between turbines in a wind farm in \cref{contrib:windpower,contrib:patent_egedal,contrib:patent_kaiser}.

Using these methods, we investigate the behavior of Bayesian structured models and discuss evaluation strategies.
In \cref{contrib:interpretable_rl,contrib:multi_modal_rl}, we apply our data association model to a reinforcement learning task and we show how a semantic decomposition of the dynamics reduces the data requirements and produces interpretable solutions.
In \cref{contrib:surrogates}, we argue why models with suboptimal marginal likelihoods can perform well in hierarchical systems such as Bayesian optimization problems.
In \cref{contrib:composition}, we present an argument for why inference schemes based on factorizations between layers cannot represent heterogeneous posteriors.


\subsubsection{Papers}
\begin{enumerate}
    \item \label[contribution]{contrib:windpower}\fullfullcite{kaiser_bayesian_2018}
    \item \label[contribution]{contrib:data_association}\fullfullcite{kaiser_data_2019}
    \item \label[contribution]{contrib:interpretable_rl}\fullfullcite{kaiser_interpretable_2019}
    \item \label[contribution]{contrib:multi_modal_rl}\fullfullcite{kaiser_bayesian_2020}
    \item \label[contribution]{contrib:surrogates}\fullfullcite{bodin_modulating_2020}
    \item \label[contribution]{contrib:composition}\fullfullcite{ustyuzhaninov_compositional_2020}
\end{enumerate}

\subsubsection{Patents}
\begin{enumerate}
    \setcounter{enumi}{6}
    \item \label[contribution]{contrib:patent_egedal}\fullfullcite{egedal_verfahren_2019}
    \item \label[contribution]{contrib:patent_kaiser}\fullfullcite{kaiser_verfahren_2019}
    \item \label[contribution]{contrib:patent_geipel}\fullfullcite{geipel_transferlernen_2020}
    \item \label[contribution]{contrib:patent_depeweg}\fullfullcite{depeweg_computer_2020}
\end{enumerate}
